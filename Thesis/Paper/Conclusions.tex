\chapter{Discussion and Conclusions}

	Based on the study and analysis performed in Chapters \ref{Chapter_2} and \ref{Chapter_3}, we can ensure that spectral methods are an excellent choice to solve problems where solutions are well behaved or are smooth, such as periodic problems or rather solutions that vanishes at bounds. 
	
	In addition, it was possible to understand some of the advantages and disadvantages that may arise in practice, such as in the cases for small viscosity coefficients studied at the end of Chapter \ref{Chapter_3} we could see that the order of convergence was lower but the solution approached the case of zero viscosity which if it has an exact solution and that could be used as a good approximation. 
	
	However, it was possible to note that the problem for these cases was due to a discontinuity or excessive changes in function, which produces strong oscillations around them and is also known as the Gibbs phenomenon, which can be reduced by choosing a adequate initial condition or consider longer intervals in the spatial variable. Although for this there are some techniques that can help improve the accuracy of these methods, one of them is to use non-uniform discretizations in the spatial variable such as Chebyshev nodes and using the known Chebyshev polynomials as bases. 
	
	In the same way with respect to the stochastic version of Burgers' equation, we can say that the methods are good choice to approximate their solutions, since apparently in the Chapter \ref{Chapter_4} they are very similar in theory and implementation in comparison with the deterministic version. 
	
	But nevertheless in general, the great advantage of these methods, whether they are implemented to the deterministic or stochastic version, is that they are easy to implement and develop very efficient computational codes in order to make numerical analysis studies or rather predictions of some phenomenon. 
	
	It would be very interesting to be able to extend these same ideas for more complex models, either for non-linear problems where their solutions are not smooth and discontinuities occur such as the case of Burgers' equation without viscosity, or even more, studying under this approach the famous problem of Navier-Stokes, whether in his classic deterministic version or his stochastic version, which could be an excellent work in the future to study.
	